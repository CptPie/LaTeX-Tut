\section{Hello World!}\subsection{Hello World!}
\begin{frame}[fragile]{Hello World!}{Grundlagen}
\begin{exampleblock}{Neue Befehle:}
\begin{itemize} %cred cturkis cpurple corange
    \item \color{cred} \textbackslash documentclass\color{black}[\color{corange}options\color{black}]\{\color{cpurple}class\color{black}\}
    \item \color{cred}\textbackslash usepackage\color{black}\{\color{cblue}package\color{black}\}
    \item \color{cturkis}\textbackslash begin\color{black}\{\color{cpurple}environment\color{black}\}
    \item \color{cturkis}\textbackslash end\color{black}\{\color{cpurple}environment\color{black}\}
\end{itemize}
\end{exampleblock}
\end{frame}
\begin{frame}{Hello World!}{Der Anfang}
Jedes \LaTeX-Dokument beginnt mit der Dokumentdefinition:\\ \vspace{1.5mm}
\color{cred} \textbackslash documentclass\color{black}[\color{corange}options\color{black}]\{\color{cpurple}class\color{black}\}\\\vspace{2.5mm}
\begin{multicols}{2}
\color{corange}options\color{black}\\
\begin{itemize}
    \item Schriftgr\"o{\ss}e in pt: z.B. \color{corange}12pt\color{black}
    \item Papierformat z.B. \color{corange}a4paper\color{black}
    \item Standartsprache z.B. \color{corange}ngerman \color{black} deutsche Formatierungen und (Trenn-)Regeln
\end{itemize}
Alle Variablen sind optional.\columnbreak\pause

\color{cpurple}class\color{black}
\begin{itemize}
    \item einfache Dokumente z.B. \color{cpurple}scrartcl, article\color{black}
    \item komplexe Dokumente z.B. \color{cpurple}scrreprt, report\color{black}
    \item B\"ucher z.B. \color{cpurple}scrbook, book\color{black}
    \item Pr\"asentationen z.B. \color{cpurple}beamer, seminar\color{black}
\end{itemize}
\end{multicols}
\normalsize
\end{frame}
\begin{frame}{Hello World!}{Packages}
    Mit dem Befehl \color{cred}\textbackslash usepackage\color{black}\{\color{cblue}package\color{black}\} kann die Basisfunktionalit\"at von \LaTeX~um weitere Funktionen erweitert werden. Die Pakete werden direkt nach der Dokumentdefinition eingebunden. Die Verwendung von Usepackages ist NICHT verpflichtend, jedoch meist nicht zu Umgehen.\\\vspace{1mm}
    Einige wichtige Usepackages:
    \begin{itemize}
        \item \color{cblue}babel \color{black}- support f\"ur zahlreiche Sprachen, mehrere Sprachen in einem Dokument
        \item \color{cblue}inputenc \color{black}- zu verwendender Zeichensatz i.d.R. \color{corange}utf8 \color{black}(als Option)
        \item \color{cblue}fontenc \color{black}- Darstellung von Sonderzeichen, i.d.R. \color{corange}T1 \color{black}(als Option)
    \end{itemize}
\end{frame}
\begin{frame}{Hello World!}{Umgebungen}
    Umgebungen sind neben der Dokumentdefinition und den Usepackages der dritte grundlegende Baustein jedes \LaTeX-Dokuments. Umgebungen werden mit \color{cturkis}\textbackslash begin\color{black}\{\color{cpurple}environment\color{black}\} ge\"offnet und mit \color{cturkis}\textbackslash end\color{black}\{\color{cpurple}environment\color{black}\} wieder geschlossen. \\\vspace{1.5mm}F\"ur den Anfang betrachten wir zun\"achst die Umgebung \color{cpurple}document\color{black}. Innerhalb der Dokumentumgebung steht der Inhalt unseres Dokuments.
\end{frame}
\begin{frame}{Hello World!}{Das \glqq Hello World!\grqq-Dokument}
    Nachdem wir die Theorie nun durch haben k\"onnen wir unser erstes Dokument schreiben.\\\vspace{2.5mm}
    \color{cred}\textbackslash documentclass\color{black}[\color{corange}10pt, a4paper, ngerman \color{black}]\{\color{cpurple}scrartcl\color{black}\}\\
    \color{cturkis}\textbackslash begin\color{black}\{\color{cpurple}document\color{black}\}\\
    Hello World!\\
    \color{cturkis}\textbackslash end\color{black}\{\color{cpurple}document\color{black}\}\\\vspace{2.5mm}
    Was dann zu diesem Ergebnis f\"uhrt:\\\vspace{2.5mm}
    \begin{center}\textrm{Hello World!}\end{center}
    
\end{frame}