\section{Tabellen}\subsection{Einfache Tabellen}
\begin{frame}{Tabellen}{\"Ubersicht}
   \begin{exampleblock}{Neue Pakete}
        \begin{multicols}{2}
        \begin{itemize}
            \item \color{cblue}longtable
            \item float \color{black}
        \end{itemize}
        \end{multicols}
   \end{exampleblock}
   \begin{block}{Neue Umgebungen}
        \begin{multicols}{3}
        \begin{itemize}
            \item \color{cpurple}tabular
            \item table
            \item longtable\color{black}
        \end{itemize}
        \end{multicols}
   \end{block}
   \begin{block}{Neue Befehle}
        \begin{multicols}{2}
        \begin{itemize}
            \item \color{cred}\textbackslash hline
            \columnbreak
            \item \color{cturkis}\textbackslash multicolumn\color{black}\{Spaltenzahl\}\\\{Spaltenausrichtung\}\{Text\}
        \end{itemize}
        \end{multicols}
   \end{block}
\end{frame}
\begin{frame}{Tabellen}{Keine sch\"one Sache \frownie}
   Tabellen in \LaTeX~sind anfangs wirklich nicht komfortabel zu erstellen. Das geht soweit, dass es Websides gibt, welche Tabellen zu \LaTeX-Code umwandeln (\small\url{http://www.tablesgenerator.com/}\normalsize).\\
   Aber wir wollen sie per Hand erstellen k\"onnen, was f\"ur einfache Tabellen eigentlich gar nicht allzu schwer ist.
\end{frame}
\begin{frame}{Aufbau von Tabellen}
   Aufbau:\\\vspace{1.5mm}
   \color{cturkis}\textbackslash begin\color{black}\{\color{cpurple}tabular\color{black}\}\{Spaltendefinitionen\}\\
   \textit{Tabelleninhalt}\\
   \color{cturkis}\textbackslash end\color{black}\{\color{cpurple}tabular\color{black}\}\\\vspace{1.5mm}
   Spaltendefinitionen:\\\vspace{1.5mm}
   \begin{tabular}{ll}
        l & linksb\"undig \\
        c & zentriert\\
        r & rechtsb\"undig\\
        p\{Breite\} & Feste Spaltenbreite\\
        \color{cred}|\color{black} & Senkrechter Strich zur Spaltentrennung
   \end{tabular}
\end{frame}
\begin{frame}{Aufbau von Tabellen}
   Aufbau:\\\vspace{1.5mm}
   \color{cturkis}\textbackslash begin\color{black}\{\color{cpurple}tabular\color{black}\}\{Spaltendefinitionen\}\\
   \textit{Tabelleninhalt}\\
   \color{cturkis}\textbackslash end\color{black}\{\color{cpurple}tabular\color{black}\}\\\vspace{1.5mm}
   Tabelleninhalt:\\\vspace{1.5mm}
   \begin{tabular}{ll}
        \color{cred}\&\color{black} & Spaltentrennung\\
        \color{cred}\textbackslash\textbackslash\color{black} & Zeilenende/Neue Zeile\\
        \color{cred}\textbackslash hline\color{black} & horizontale Linie\\
        \color{cturkis}\textbackslash multicolumn\color{black}\{Spaltenzahl\} & Verbindet beliebig viele Spalten\\\{Spaltenausrichtung\}\{Text\}&miteinander
   \end{tabular}
\end{frame}
\begin{frame}{Eine Einfache Tabelle}
   \color{cturkis}\textbackslash begin\color{black}\{tabular\}\{c|p\{30mm\}|lr||c\}\\
   \textbackslash multicolumn\{5\}\{c\}\{Ergebnistabelle\}\\
   \color{cred}\textbackslash hline \textbackslash hline  \textbackslash\textbackslash\color{black}\\
   Nummer \& Ort \& Spieler 1 \& Spieler 2 \& Ergebnis \color{cred}\textbackslash\textbackslash~\textbackslash hline\color{black}\\ 
   1 \& Bamberg \& Fuchs \& Bauer \& 23:10 \color{cred}\textbackslash\textbackslash~\textbackslash hline\color{black}\\
   2 \& tba \& Wolf \& Baier \color{cred}\textbackslash\textbackslash~\textbackslash hline\color{black}\\
   \color{cturkis}\textbackslash end\color{black}\{tabular\}\\\vspace{1.5mm}
   F\"uhrt beispielsweise zu folgender Tabelle:\\\vspace{1.5mm}
   \begin{tabular}{c|p{30mm}|lr||c}
    \multicolumn{5}{c}{Ergebnistabelle}\\
    \hline \hline
    Nummer & Ort & Spieler 1 & Spieler 2 & Ergebnis \\\hline
    1 & Bamberg & Fuchs & Bauer & 23:10 \\ \hline
    2 & tba & Wolf & Baier \\\hline
   \end{tabular}
\end{frame}
\begin{frame}{Zeilenumbr\"uche \& lange Tabellen}
Bei festgelegter Spaltenbreite (p\{Breite\}) bricht \LaTeX~ automatisch die Zeilen um.\\
\begin{center}
\begin{tabular}{|p{20mm}|c|}\hline
    Langer Text &  bla\\\hline
    Sehr langer Text & blub \\\hline
\end{tabular}
\end{center}
\vspace{3mm}Au\ss erdem k\"onnen wir anstatt der \color{cpurple}tabular\color{black}-Umgebung auch die \color{cpurple}longtable\color{black}-Umgebung nutzen, welche es erm\"oglicht, Tabellen \"uber mehrere Seiten fortzuf\"uhren.\\\vspace{2mm}
\color{cturkis}\textbackslash begin\color{black}\{\color{cpurple}longtable\color{black}\}\{Spaltendefinitionen\}\\
   \textit{Tabelleninhalt}\\
   \color{cturkis}\textbackslash end\color{black}\{\color{cpurple}longtable\color{black}\}\
\end{frame}
\subsection{Erweiterte Funktionen}
\begin{frame}{Erweiterte Funktionen}
Tabellen k\"onnen mit Hilfe der \color{cpurple}table\color{black}-Umgebung und dem Usepackage \color{cturkis}float\color{black}~um einige Funktionalit\"aten erweitert werden.\\\vspace{2mm}
Dazu erstellen wir zun\"achst eine \glqq normale\grqq~Tabelle. \uncover<2->{Und erweitern sie um diese Ausdr\"ucke.}\\\vspace{3mm}\small

\uncover<2->{
\color{cturkis}\textbackslash begin\color{black}\{table\}[H]\\
\color{cturkis}\textbackslash centering\\
}
~~~~~~~~\color{cturkis}\textbackslash begin\color{black}\{tabular\}\{|p\{20mm\}|c|\}\color{cred}\textbackslash hline \color{black}\\
~~~~~~~~Langer Text \color{cred}\&~\color{black} bla \color{cred}\textbackslash\textbackslash\textbackslash hline\\\color{black}
~~~~~~~~Sehr langer Text \color{cred}\&~\color{black} blub \color{cred}\textbackslash\textbackslash\textbackslash hline\color{black}\\
~~~~~~~~\color{cturkis}\textbackslash end\color{black}\{tabular\}\\
\uncover<2->{
\color{cturkis}\textbackslash caption\color{black}\{Unsere Tabelle!\}\\
\color{cblue}\textbackslash label\color{black}\{tab:Tabelle1\}\\
\color{cturkis}\textbackslash end \color{black}\{table\}
}\normalsize
\end{frame}
\begin{frame}{Erweiterte Funktionen}
Also fangen wir mal oben an. Die \color{cpurple}table\color{black}-Umgebung wird f\"ur die erweiterten Funktionen ben\"otigt (damit \LaTeX~wei\ss, worauf sich die Befehle beziehen).\\
In den eckigen Klammern wird die Position der Tabelle angegeben.\\
\begin{itemize}
    \item h - \glqq here\grqq; hier, wenns geht
    \item t - \glqq top\grqq; am Anfang der Seite
    \item b - \glqq bottom\grqq; am Ende der Seite
    \item p - \glqq page\grqq; auf einer eigenen Seite
    \item H - \glqq Here\grqq; hier, auf jeden Fall (ben\"otigt Usepackage \color{cblue}float\color{black})
\end{itemize}
\color{cturkis}\textbackslash caption \color{black}f\"ugt einen Untertitel hinzu, mit \color{cblue}\textbackslash label~\color{black} geben wir der Tabelle eine einzigartige Bezeichnung (dazu sp\"ater mehr).\\
\vspace{3mm}\color{cpurple}longtable\color{black}~ben\"otigt die \color{cpurple}table\color{black}-Umgebung~NICHT, longtable ist eine eigene Umgebung.
\end{frame}