\section{Einf\"uhrung}\subsection{Einf\"uhrung}
\begin{frame}{Einf\"uhrung}{Sinn - Unsinn - Wahnsinn}
   \begin{multicols}{3}
   Sinn:
   \begin{itemize}
       \item wissenschaftliche Arbeiten \& Mitschriften
       \item B\"ucher
       \item Offizielles (Bewerbungen, etc.)
   \end{itemize}
   \columnbreak\pause
 
   Unsinn:
   \begin{itemize}
       \item private Briefe
       \item Geburtstags- einladungen
       \item Getr\"ankekarten
   \end{itemize}
   \columnbreak\pause
   
   Wahnsinn:
   \begin{itemize}
       \item Einkaufszettel
       \item Brainstorming
       \item ...
   \end{itemize}
   \end{multicols} 
\end{frame}
\begin{frame}{Einf\"uhrung}{Grundlagen}
    \begin{itemize}
        \item kein WYSIWYG\pause
        \item Trennung zwischen Layout und Inhalt\pause
        \item Layout wird genau definiert\pause
        \begin{itemize}
            \item durch Code!\pause
        \end{itemize}
        \item Standard f\"ur wissenschaftliche Arbeiten in MINT
    \end{itemize}
\end{frame}
\begin{frame}{Einf\"uhrung}{\LaTeX: Pro \& Contra}
    \begin{multicols}{2}
        Pro:\\
        \begin{itemize}
            \item automatisches Layout / einfache Nutzung von Layouts
            \item einfaches verteiltes Arbeiten
            \item System unabh\"angiges Dateiformat (Output als PDF/dvi)
            \item dynamisches Kapitel-, Abbildungs- und Referenzverzeichnis
        \end{itemize}\columnbreak\pause
        
        Contra:\\
        \begin{itemize}
            \item steile Lernkurve (falls man noch NIE programmiert hat)
            \item kann schnell un\"ubersichtlich werden
            \item Datei muss kompiliert werden um \"Anderungen zu sehen
            \item Compiler muss installiert sein bzw. Nutzung von Onlinekompilern
        \end{itemize}
    \end{multicols}
\end{frame}
\begin{frame}[fragile]{Einf\"uhrung}{Was braucht man?}
    \Large\LaTeX-Compiler\normalsize\\
    \begin{multicols}{2}
        Windows:\\
        \begin{itemize}
            \item MikTex\\
            \url{http://www.miktex.org}
            \item ProTeXt\\
            \url{http://www.tug.org/protext}
        \end{itemize}
        \columnbreak
        
        Unix:
        \begin{itemize}
            \item \textbf{Linux:} TeXLive\\
                \begin{verbatim}
sudo apt-get install
texlive-full
                \end{verbatim}
            \item \textbf{MacOS:} MacTeX\\
            \url{http://www.tug.org/mactex/}
        \end{itemize}
    \end{multicols}
\end{frame}
\begin{frame}{Einf\"uhrung}{Was braucht man?}
    \Large\LaTeX-Editoren\normalsize\\\vspace{.5cm}
    generell kann man mit jedem Editor \LaTeX~schreiben.\\
    Aber es gibt spezialisierte Editoren:
    \begin{itemize}
        \item TeXstudio\\
        \url{http://texstudio.sourceforge.net}
        \item TeXlipse\\
        \url{http://texlipse.sourceforge.net}
    \end{itemize}
\end{frame}
\begin{frame}{Einf\"uhrung}{Was braucht man?}
    \Large\LaTeX-Onlineeditoren\normalsize\\\vspace{.5cm}
    Kooperation mit mehreren Personen parallel an einem Dokument.
    \begin{itemize}
        \item Sharelatex\\
        \url{https://de.sharelatex.com/}
        \item Overleaf\\
        \url{https://www.overleaf.com/}
    \end{itemize}
\end{frame}