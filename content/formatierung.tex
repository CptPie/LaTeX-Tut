\section{Formatierung und Layout}
\begin{frame}{\"Ubersicht}
\begin{exampleblock}{Neue Umgebungen:}
\begin{multicols}{3}
\begin{itemize}
    \item \color{cpurple}enumerate
\end{itemize}\columnbreak
\begin{itemize}
    \item \color{cpurple}itemize
\end{itemize}\columnbreak
\begin{itemize}
    \item \color{cpurple}tabbing
\end{itemize}
\end{multicols}\color{black}
\end{exampleblock}
    \begin{block}{Neue Befehle:}
        \begin{multicols}{2}
        \begin{itemize}
            \item \color{cturkis}\textbackslash textbf\color{black}\{Text\}
            \item \color{cturkis}\textbackslash textit\color{black}\{Text\}
            \item \color{cturkis}\textbackslash underline\color{black}\{Text\}
            \item \color{cred}\textbackslash\textbackslash\color{black}
            \item\color{cturkis}\textbackslash hspace\color{black}\{\color{corange}L\"ange\color{black}\}
            \item\color{cturkis}\textbackslash vspace\color{black}\{\color{corange}L\"ange\color{black}\}
            \item \color{cred}\textbackslash item \color{black} $\hdots$
        \end{itemize}\columnbreak
        \begin{itemize}
            \item \color{cturkis}\textbackslash section\color{black}\{Titel\}
            \item \color{cturkis}\textbackslash subsection\color{black}\{Titel\}
            \item \color{cturkis}\textbackslash subsubsection\color{black}\{Titel\}
            \item \color{cturkis}\textbackslash tableofcontents
            \item \color{cred}\textbackslash= \textbackslash>
            \item \color{cred}\textbackslash kill
        \end{itemize}
        \end{multicols}
        
    \end{block}
\end{frame}
\subsection{Einfache Formatierungen}
\begin{frame}{Fett, kursiv und unterstreichen}
    Die g\"angigen Formatierungen \textbf{fett}, \textit{kursiv} und \underline{unterstrichen} werden in \LaTeX~durch Befehle realisiert.
    \begin{itemize}
        \item \textbf{fett} - \color{cturkis}\textbackslash textbf\color{black}\{fetter Text\}
        \item \textit{kursiv} -\color{cturkis}\textbackslash textit\color{black}\{kursiver Text\}
        \item \underline{unterstrichen} -\color{cturkis}\textbackslash underline\color{black}\{unterstrichener Text\}
    \end{itemize}\pause
    Diese Formatierungen sind auch kumulativ verwendbar, so f\"uhrt \color{cturkis}\textbackslash textbf\color{black}\{\color{cturkis}\textbackslash textit\color{black}\{Hallo!\}\} zu diesem Ergebnis:
    \begin{center}
    \textrm{\textbf{\textit{Hallo!}}}
    \end{center}\pause
    Hier ist noch zu bemerken, dass bei verschiedenen Schriftarten die kumulative Verwendung nicht m\"oglich sein kann. (z.B. auch bei der f\"ur die Pr\"asentation verwendeten \smiley)
\end{frame}
\begin{frame}[fragile]{Textlayout}
    Einen einfachen Zeilenumbruch erzeugt man mit \color{cred}\textbackslash\textbackslash\color{black}. Eine Leerzeile entweder mit 2 Zeilenumbr\"uchen (Compiler kann Warnings werfen) oder durch eine Leerzeile im Dokument.
    \textrm{\begin{multicols}{2}
        Nach diesem Satz kommt ein Zeilenumbruch.\color{cred}\textbackslash\textbackslash\color{black}\\
        Und hiernach eine Leerzeile.\color{cred}\textbackslash\textbackslash\textbackslash\textbackslash\color{black}\\
        Oder?\\
        \columnbreak
        Nach diesem Satz kommt ein Zeilenumbruch.\\
        Und hiernach eine Leerzeile.\\
        \vspace{\baselineskip}
        Oder?
    \end{multicols}}
    Nat\"urlich kann man in \LaTeX~auch einfach eine neue Seite Anfangen, z.B. mit \color{cturkis}\textbackslash newpage\color{black}.
\end{frame}
\subsection{Wir nehmen den Text und schieben ihn wo anderst hin!}
\begin{frame}{Einr\"uckungen und Whitespace}
    Mit dem Befehl \color{cturkis}\textbackslash hspace\color{black}\{\color{corange}L\"ange\color{black}\} k\"onnen Einr\"uckungen erstellt werden. Als Eingaben werden die g\"angigsten L\"angeneinheiten akzeptiert von \color{corange}pt \color{black} \"uber  \color{corange}cm \color{black} zu \color{corange}in \color{black} und noch viele weitere. Beispielsweise bewirkt \color{cturkis}\textbackslash hspace\{\color{corange}1.5cm\color{black}\} diesen \hspace{1.5cm} Abstand.\\
    Analog dazu bewirkt \color{cturkis}\textbackslash vspace\color{black}\{\color{corange}1.5cm\color{black}\} diesen \\\vspace{1.5cm} Abstand nach einem \textbf{Absatz}.
\end{frame}
\begin{frame}{Tabbing}
    \begin{tabbing}
    Nat\"urlich kann man auch in \LaTeX~Tabstopps setzen, wobei das \\umst\"andlicher als in \glqq normalen\grqq~Textverarbeitungsprogrammen ist.\\
    Zun\"achst erstellt man eine \color{cpurple}tabbing\color{black}-Umgebung und erstellt anschlie{\ss}end \\eine Tabbing-Zeile:\\\vspace{1.5mm}
    XXXX\color{cred}\textbackslash=\color{black}XXXXXX\color{cred}\textbackslash=\textbackslash kill\color{black}\\
    XXXX\=XXXXXX\=\kill
    \color{cred}\textbackslash= \color{black} steht hierbei f\"ur den Tabstopp und \color{cred}\textbackslash kill \color{black} bewirkt, dass diese Zeile \\nicht ausgegeben wird.
    Hierdurch werden Tabstopps im Abstand von \\4x X bzw 10x X in jeder Zeile gesetzt. \\
    Ein Tabstopp kann mit \color{cred}\textbackslash> \color{black}angesteuert werden.\\
    Hier\\
    \>ist\\
    \>\>getabbt worden.
    \end{tabbing}
\end{frame}
\subsection{Struktur im Text}\label{sec:toc}
\begin{frame}{\"Uberschriften}
    \LaTeX~bietet umfassende M\"oglichkeiten Texte zu strukturieren. 
    \begin{itemize}
        \item \color{cturkis}\textbackslash section\color{black}\{Titel\} erstellt eine neue \"Uberschrift erster Ordnung 
        \item \color{cturkis}\textbackslash subsection\color{black}\{Titel\} erstellt eine \"Uberschrift zweiter Ordnung
        \item \color{cturkis}\textbackslash subsubsection\color{black}\{Titel\} ja ratet mal ...
        \item von Haus aus unterst\"utzt \LaTeX~nur 3 Gliederungsebenen (online gibts Hilfe \smiley)
    \end{itemize}
    Da wir gerade von einer Gliederung reden, diese kann mit \color{cturkis}\textbackslash tableofcontents \color{black}erstellt werden, wobei automatisch alle \"Uberschriften in der richtigen Reihenfolge und Nummerierung aufgef\"uhrt werden.
\end{frame}
\begin{frame}[fragile]{Nummerierung und Aufz\"ahlung}
\begin{multicols}{2}
Nummerierung
\scriptsize
\begin{tabbing}
XXX\=XXXX\=XX\=\kill
\color{cturkis}\textbackslash begin\{\color{cpurple}enumerate\color{black}\}\\
\>\color{cred}\textbackslash item \color{black}Item 1\\
\>\color{cred}\textbackslash item \color{black}Item 2\\
\>\color{cred}\textbackslash item \color{cturkis}\textbackslash begin \color{black}\{\color{cpurple}enumerate\color{black}\}\\
\>\>\>\color{cred}\textbackslash item \color{black}Item 3.1\\
\>\>\>\color{cred}\textbackslash item \color{black}Item 3.1\\
\>\>\color{cturkis}\textbackslash end \color{black}\{\color{cpurple}enumerate\color{black}\}\\
\>\color{cred}\textbackslash item \color{black}Item 4\\
\color{cturkis}\textbackslash end\{\color{cpurple}enumerate\color{black}\}\\
\end{tabbing}
\normalsize
\begin{enumerate}
    \item Item 1
    \item Item 2
    \item \begin{enumerate}
            \item Item 3.1
            \item Item 3.2
          \end{enumerate}
    \item Item 4
\end{enumerate}
\columnbreak
Aufz\"ahlung
\scriptsize
\begin{tabbing}
XXX\=XXXX\=XX\=\kill
\color{cturkis}\textbackslash begin\{\color{cpurple}itemize\color{black}\}\\
\>\color{cred}\textbackslash item \color{black}Item 1\\
\>\color{cred}\textbackslash item \color{black}Item 2\\
\>\color{cred}\textbackslash item \color{cturkis}\textbackslash begin \color{black}\{\color{cpurple}itemize\color{black}\}\\
\>\>\>\color{cred}\textbackslash item \color{black}Item 3.1\\
\>\>\>\color{cred}\textbackslash item \color{black}Item 3.1\\
\>\>\color{cturkis}\textbackslash end \color{black}\{\color{cpurple}itemize\color{black}\}\\
\>\color{cred}\textbackslash item \color{black}Item 4\\
\color{cturkis}\textbackslash end\{\color{cpurple}itemize\color{black}\}\\
\end{tabbing}
\normalsize
\begin{itemize}
    \item Item 1
    \item Item 2
    \item \begin{itemize}
            \item Item 3.1
            \item Item 3.2
          \end{itemize}
    \item Item 4
\end{itemize}
\end{multicols}
\end{frame}
\subsection{Sonderzeichen}
\begin{frame}{Sonderzeichen}
    Wie jede andere (Programmier-) Sprache, besitzt auch \LaTeX~ einige Sonderzeichen, welche f\"ur Befehle reserviert sind und daher bei Verwendung escaped werden m\"ussen.
    \begin{table}[]
        \centering
        \begin{tabular}{|c|c|c|}\hline
            Zeichen & Funktion & Eingabe zur Erzeugung \\\hline
            \% & Kommentar & \textbackslash\%  \\\hline
            \$ & Formelumgebung & \textbackslash\$ \\\hline
            \& & Spaltentrennung (Tabellen) & \textbackslash\& \\\hline
            \{ \} & Befehlsparameter & \textbackslash\{ \textbackslash\} \\\hline
            \# & - & \textbackslash\#\\\hline
            \textbackslash & Befehlsbeginn & \textbackslash textbackslash\\\hline
            \^~\_ & hoch-/tiefstellen & \textbackslash\^~ \textbackslash\_\\\hline
            \~{} & Leerzeichen erzwingen &  \textbackslash\~{}\{\}\\\hline
        \end{tabular}
        \caption{Sonderzeichen}
    \end{table}
\end{frame}
\begin{frame}{Sonderzeichen}
    Weitere Sonderzeichen: (falls Taste nicht verf\"ugbar)
    \begin{table}[]
        \centering
        \begin{tabular}{|c|c|}\hline
            Zeichen & Eingabe \\\hline
            \"A \"a (genauso wie \"o und \"u)& \textbackslash \grqq A  \textbackslash \grqq a  \\\hline
            \ss & \textbackslash ss bzw \{\textbackslash ss\}\\\hline
            \pounds & \textbackslash pounds \\\hline
            \euro & \textbackslash euro (ben\"otigt package \color{cblue}eurosym\color{black})\\\hline
            \glqq~~~~\grqq & \textbackslash glqq~~~~\textbackslash grqq (ben\"otigt package \color{cblue}babel\color{black})\\\hline
            \flqq~~~~ \frqq & \textbackslash flqq~~~~\textbackslash frqq (ben\"otigt package \color{cblue}babel\color{black})\\\hline
        \end{tabular}
        \caption{Weitere Sonderzeichen}
    \end{table}
\end{frame}